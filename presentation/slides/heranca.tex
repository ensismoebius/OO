\begin{frame}
	\frametitle{Herança}
	\begin{itemize}
		\item Permite que uma classe herde atributos e métodos de outra classe
		\item Promove a reutilização de código
	\end{itemize}
\end{frame}

\begin{frame}[fragile]
	\frametitle{Exemplo em Java: Herança}
	\begin{verbatim}
		public class Animal {
			void fazerSom() {
				System.out.println("Barulho genérico");
			}
		}
		
		public class Cachorro extends Animal {
			void fazerSom() {
				System.out.println("Latido");
			}
		}
	\end{verbatim}
\end{frame}

\begin{frame}
	\frametitle{Polimorfismo}
	\begin{itemize}
		\item Permite que objetos de diferentes classes sejam tratados de forma uniforme
		\item Pode ocorrer através de sobrecarga ou sobreposição de métodos
	\end{itemize}
\end{frame}

\begin{frame}[fragile]
	\frametitle{Exemplo em Java: Polimorfismo}
	\begin{verbatim}
		public class Calculadora {
			int somar(int a, int b) {
				return a + b;
			}
			
			double somar(double a, double b) {
				return a + b;
			}
		}
	\end{verbatim}
\end{frame}

\begin{frame}
	\frametitle{Conclusão}
	\begin{itemize}
		\item Orientação a Objetos é um paradigma poderoso para desenvolvimento de software
		\item Java é uma linguagem amplamente utilizada que suporta os princípios de OOP
	\end{itemize}
\end{frame}