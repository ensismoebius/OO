\begin{frame}
	\frametitle{Herança}
	\begin{itemize}
		\item Permite que uma classe herde atributos e métodos de outra classe
		\item Promove a reutilização de código
	\end{itemize}
\end{frame}

\begin{frame}
	\frametitle{Polimorfismo}
	\begin{itemize}
		\item Permite que métodos seja sobrecarregados
	\end{itemize}
\end{frame}

\begin{frame}[fragile]
	\frametitle{Exemplo em Java: Moradia.java}
	\begin{lstlisting}[language=java]
public class Moradia {
	public String cor;
	public int qtdeDePisos;
	protected int anoDeConstrucao;
	private String cnpjDoCapitalistaSafado;
	public int getAnoDeConstrucao() {
		return this.anoDeConstrucao;
	}
	public boolean setAnoDeConstrucao(int anoDeConstrucao) {
		if (anoDeConstrucao < 0) return false;
		this.anoDeConstrucao = anoDeConstrucao;
		return true;
	}
	public void abrirPorta() {
		System.out.println("Nheeeeeeee.....");
	}
	public void exibeInformacoes() {
		System.out.println("Ano:" + anoDeConstrucao + " Cor:" + cor + " Pisos:" + qtdeDePisos + " Cnpj: " + this.cnpjDoCapitalistaSafado);
	}
	public String getCnpjDoCapitalistaSafado() {
		return cnpjDoCapitalistaSafado;
	}
	public boolean setCnpjDoCapitalistaSafado(String cnpjDoCapitalistaSafado) {
		if (cnpjDoCapitalistaSafado.length() == 14) {
			this.cnpjDoCapitalistaSafado = cnpjDoCapitalistaSafado;
			return true;
		}
		return false;
	}
}
	\end{lstlisting}
\end{frame}

\begin{frame}[fragile]
	\frametitle{Exemplo em Java: Herança e polimorfismo - Casa.java}
	\begin{lstlisting}[language=java]
public class Casa extends Moradia {
	
	// Isso aqui eh um metodo construtor
	public Casa() {
		
		// Isso aqui vai da erro
		// this.cnpjDoCapitalistaSafado
		
		// A propriedade cnpjDoCapitalistaSafado eh privada 
		// portanto nao pode ser herdada. 
		// Mas setCnpjDoCapitalistaSafado eh publico
		// e herdavel
		this.setCnpjDoCapitalistaSafado("00000000000000");
	}
	
	// Isso aqui eh uma annotation indicando que o 
	// metodo esta sendo sobrecarregado
	@Override
	public void abrirPorta() {
		System.out.println("Abre a porta mariquinha");
	}
	
}
	\end{lstlisting}
\end{frame}

\begin{frame}[fragile]
	\frametitle{Exemplo em Java: Herança e polimorfismo - OO.java}
	\begin{lstlisting}[language=java]
public class OO {
	public static void main(String[] args) {
		
		Moradia minhaCasa = new Moradia();
		minhaCasa.setAnoDeConstrucao(2024);
		minhaCasa.cor = "Rosa";
		minhaCasa.qtdeDePisos = 1;
		minhaCasa.abrirPorta();
		minhaCasa.exibeInformacoes();
		
		// Os mesmos metodos estao sendo chamados mas os resultados sao diferentes gracas a heranca e ao polimorfismo
		Casa casaAmarela = new Casa();
		casaAmarela.setAnoDeConstrucao(2017);
		casaAmarela.cor = "Amarela";
		casaAmarela.qtdeDePisos = 2;
		casaAmarela.abrirPorta();
		casaAmarela.exibeInformacoes();
		
		// Os mesmos metodos estao sendo chamados mas os resultados sao diferentes gracas a heranca e ao polimorfismo e mais, tambem ha um metodo a mais: subirDeElevador()
		Apartamento meuAp = new Apartamento();
		meuAp.setAnoDeConstrucao(2017);
		meuAp.cor = "Branco";
		meuAp.qtdeDePisos = 1;
		meuAp.abrirPorta();
		meuAp.subirDeElevador();
		meuAp.exibeInformacoes();
	}
}
	\end{lstlisting}
\end{frame}

\begin{frame}
	\frametitle{Exercício}
	\par Implemente três classes:
	A primeira deve conter o método principal, no método principal se deve instanciar a segunda e a terceira classe.
	\par A Segunda classe deve ser uma classe comum que não estende qualquer  outra classe.
	\par A Terceira classe deve estender a segunda, sobrescrever algum de seus métodos, e implementar um método próprio. 
	\par Use o exemplo informado nos slides anteriores como base. 
\end{frame}