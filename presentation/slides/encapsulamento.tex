\begin{frame}
	\frametitle{Encapsulamento}
	\par Antes de entrar mais profundamente no assunto de encapsulamento precisamos entender o que são as visibilidades das propriedades e métodos em orientação a objetos.
	
	\begin{itemize}
		\item \textbf{Visibilidade Pública}: indica que aquele atributo ou método será acessível para outros objetos um atributo público também é herdável (veremos sobre herança mais adiante)
		\item V\textbf{isibilidade privada}: indica que o atributo ou método não será acessível para outros objetos um atributo privado não é herdável.
		\item \textbf{Visibilidade Protegida}: indica que o atributo ou método protegido também não é acessível a outros objetos, no entanto, eles são herdáveis. 
	\end{itemize}
\end{frame}

\begin{frame}
	\frametitle{Encapsulamento}
	
	\par Em orientação a objetos, dependendo do estilo de programação da programadora ou do programador, às vezes é desejável \textbf{"esconder"} alguns atributos e métodos para assim \textbf{evitar algum acesso ou modificação acidentais}. \newline
	
	\par Isso é válido principalmente para os atributos pois alguns podem necessitar de algum tipo de \textbf{validação} antes de serem modificados. Se quisermos fazer isso  é necessário definir sua visibilidade como \textbf{privada}, dessa forma, não haverão modificações indesejadas.No entanto isso deixa o atributo inacessível impossibilitando qualquer modificação que seja necessária, para resolver esse problema implementamos o que se convencionou dizer chamar como \textbf{“getters”} e \textbf{“setters”} que são métodos públicos que terão o papel de validar os dados de entrada e alterar os atributos (setters) ou ler os valores que os mesmos contém (getters). 
\end{frame}

\begin{frame}[fragile]
	\frametitle{Encapsulamento}
	\begin{lstlisting}[language=java]
		public class Casa {
			
			private String cep; // Propriedade encapsulada
			private int altura; // Propriedade encapsulada


			// Getters						
			public String getCep() {
				return cep;
			}
			
			public int getAltura() {
				return altura;
			}
			
			// Setters
			public void setCep(String cep) {
				this.cep = cep;
			}
			
			public void setAltura(int altura) {
				this.altura = altura;
			}
			
		}
	\end{lstlisting}
\end{frame}

\begin{frame}[fragile]
	\frametitle{Exemplo em Java: Encapsulamento}
	\begin{lstlisting}[language=java]
		public class ContaBancaria {
			private double saldo;
			
			public double getSaldo() {
				return saldo;
			}
			
			public void depositar(double valor) {
				saldo += valor;
			}
		}
	\end{lstlisting}
\end{frame}

\begin{frame}
	\frametitle{Exercícios}
	\par Faça um programa que contenha duas classes a primeira classe deve conter o método principal e ele deve instanciar a segunda classe, já a segunda classe deve conter três atributos um público e dois privados. Implemente o encapsulamento desses atributos privados.
	
\end{frame}