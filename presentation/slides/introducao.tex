\begin{frame}
	\frametitle{Competências, Habilidades}
	
	\par 1 - Projetar aplicativos móveis, selecionando linguagens de programação e ambientes de desenvolvimento.
	
	\begin{itemize}
		\item[\textbf{1.1}] Codificar aplicativos para dispositivos móveis.
		\item[\textbf{1.2}] Utilizar ambientes de desenvolvimento de software para dispositivos móveis.
		\item[\textbf{1.3}] Construir interface gráfica para dispositivos móveis.
		\item[\textbf{1.4}] Utilizar recursos de aparelhos celulares e tablets.
	\end{itemize}
\end{frame}
\begin{frame}[allowframebreaks]
	\frametitle{Bases tecnológicas}
	
	\par \textbf{Desenvolvimento de aplicativos para dispositivos móveis}
	
	\begin{itemize}
		\item Arquiteturas e plataformas de mercado;
		\item Modelos de desenvolvimento:
		\begin{itemize}
			\item Nativo;
			\item Nativo multiplataforma;
			\item Híbrido.
		\end{itemize}
		\item Lojas de aplicativos.
	\end{itemize}
	
	\par \textbf{Conceitos do Modelo e Plataforma de Desenvolvimento}
	
	\begin{itemize}
		\item Filosofia e arquitetura;
		\item Fundamentos da plataforma;
		\item Ciclo de vida e processo de desenvolvimento;
		\item Ferramentas (SDK, IDE/CLI, emuladores entre outros);
		\item Configuração do aplicativo e permissões.
	\end{itemize}
	
	\par \textbf{Interface com o Usuário}
	
	\begin{itemize}
		\item Layouts e estilização;
		\item Componentes (texto, botões, imagens, listas, componentes para entrada de dados);
		\item Splash, diálogos e notificações;
		\item Navegação e roteamento.
	\end{itemize}
	
	\par \textbf{Armazenamento de Dados no Lado Cliente}
	
	\begin{itemize}
		\item Gerenciamento de estado dos componentes;
		\item Armazenamento de dados offline.
	\end{itemize}
\end{frame}