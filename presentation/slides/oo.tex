\begin{frame}
	\frametitle{Classes e Objetos}
	
	\par Uma \textbf{classe} é um modelo para criar objetos que define atributos e métodos. Uma classe também pode ser definida como uma especificação de alguma coisa.
	
	\par Um \textbf{atributo} é uma \textbf{característica} cujo valor pode variar de objeto para objeto.
	
	\par Um \textbf{método} é uma \textbf{ação} que \textbf{pode ou não} variar de objeto para objeto dependendo se tais métodos levam ou não em consideração os valores dos atributos.
	
	\par Já o \textbf{objeto} é aquilo que existe \textbf{real ou virtualmente} segundo a especificação da classe.

	\par \textbf{Exemplos}:
	\par O projeto de uma casa pode ser considerado uma classe já que especifica todas as coisas que a casa fará e todas as propriedades (materiais, isolamento, pisos, etc.) que a casa terá e que, caso desejem, as pessoas que morarão nessa casa poderão ou não modificar. Já o objeto é casa depois de construída de acordo com o projeto: Ela foi construída segundo o seu projeto original, ou seja, segundo sua classe.
\end{frame}

\begin{frame}[fragile]
	\frametitle{Exemplo em Java: Classe e Objeto}
	\par Arquivo \textit{Casa.java}
	\begin{lstlisting}[language=Java]
		\label{lst:code1}
package org.dedira.oo;

public class Casa {
	public String cor;
	public int qtdeDePisos;
	private int anoDeConstrucao;
	
	public void abrirPorta() {
		System.out.println("Nheeeeeeee.....");
	}

	public int getAnoDeConstrucao() {
		return this.anoDeConstrucao;
	}
	
	public boolean setAnoDeConstrucao(int anoDeConstrucao) {
		
		if (anoDeConstrucao < 0) {
			return false;
		}
		
		this.anoDeConstrucao = anoDeConstrucao;
		return true;
	}
	
	public void exibeInformacoes() {
		System.out.println("Ano:" + anoDeConstrucao + " Cor:" + cor + " Pisos:" + qtdeDePisos);
	}
}
	\end{lstlisting}
\end{frame}

\begin{frame}[fragile]
	\frametitle{Exemplo em Java: Classe e Objeto}
	\par Arquivo \textit{OO.java}
	\begin{lstlisting}[language=Java]
package org.dedira.oo;

public class OO {
	
	public static void main(String[] args) {
		
		Casa minhaCasa = new Casa();
		minhaCasa.setAnoDeConstrucao(2024);
		minhaCasa.cor = "Rosa";
		minhaCasa.qtdeDePisos = 1;
		
		minhaCasa.exibeInformacoes();
	}
}
	\end{lstlisting}
\end{frame}

\begin{frame}
	\frametitle{Exercícios}
	\par Crie um novo programa que represente a entidade trabalhadore. Essa entidade deve ter 10 atributos e 5 acões. Pelo menos 4 das 5 ações devem levar em consideração ou usar pelo menos 5 atributos cada uma.
	
	\par O programa deve rodar devidamente.
\end{frame}

\begin{frame}[fragile]
	\frametitle{Definições básicas}
	
	\par Um programa em Java é composto por classes que são compostas por \textbf{propriedades} e \textbf{métodos}.

\end{frame}

\begin{frame}[fragile]
\frametitle{Definições básicas}
	
	\par Uma \textbf{classe} pode ser definida como um modelo que determina como o objeto se comporta e quais suas características.
	\begin{lstlisting}[language=Java]
	// Linhas que comecam com "//" sao comentarios
	// Um comentario nao e executado, serve apenas 
	// como uma manual de instrucoes para facilitar
	// o entendimento do codigo
	
	// O corpo da classe e tudo aquilo 
	// que esta dentro das chaves
	// A linha abaixo define a assinatura de uma classe. 
	public class OO {
		// O corpo da classe fica aqui	
	}
	\end{lstlisting}

\end{frame}

\begin{frame}[fragile]
\frametitle{Definições básicas}
	
	
	\par Um \textbf{método} é uma ação que o objeto pode realizar. Dentro do contexto da programação um método é uma função ou procedimento, ou seja, um bloco de código que tem parâmetros e que pode retornar ou não algum valor.
	\begin{lstlisting}[language=Java] 
		public class OO {
			
			// Um metodo similar a um procedimento.Perceba que ele retorna "void", ou seja, um valor vazio
			public void abrirPorta() {
				System.out.println("Nheeeeeeee.....");
			}
			
			// Um metodo similar a uma funcao. Esse retorna "int", ou seja, um numero inteiro
			public int getAnoDeConstrucao() {
				return 10;
			}
			
			// Esse eh um metodo que retorna um "boolean", ou seja, um valor verdadeiro ou falso.
			// Perceba que esse metodo eh uma funcao que recebe parametros
			public boolean setAnoDeConstrucao(int anoDeConstrucao) {
				
				if (anoDeConstrucao < 0) {
					return false;
				}
				
				return true;
			}
		}
	\end{lstlisting}

\end{frame}

\begin{frame}[fragile]
\frametitle{Definições básicas}
	
	\par Uma \textbf{propriedade} é uma variável que é visível para todo o objeto, ou seja, todos os métodos e outras propriedades \textbf{internas} do objeto podem enxergar essa variável.
	\begin{lstlisting}[language=Java]
		public class Casa {
			// Essa eh uma propriedade do tipo "String", ou seja, um texto
			public String cor;
			// Essa tambem eh uma propriedade do tipo "int", ou seja, um numero inteiro
			public int qtdeDePisos;
		}
	\end{lstlisting}
	
\end{frame}

\begin{frame}
	\frametitle{Tipos Primitivos}
	\par Os tipos primitivos em Java são aqueles que representam valores simples. Aqui estão os principais tipos primitivos:0

	\begin{itemize}
		\item \texttt{byte}: representa números inteiros de 8 bits.
		\item \texttt{short}: representa números inteiros de 16 bits.
		\item \texttt{int}: representa números inteiros de 32 bits.
		\item \texttt{long}: representa números inteiros de 64 bits.
		\item \texttt{float}: representa números de ponto flutuante de precisão simples.
		\item \texttt{double}: representa números de ponto flutuante de dupla precisão.
		\item \texttt{char}: representa um único caractere Unicode.
		\item \texttt{boolean}: representa valores lógicos verdadeiro ou falso.
	\end{itemize}
\end{frame}

\begin{frame}
	\frametitle{Tipos de referência}
	\subsection{Tipos de Referência}
	
	\par Os tipos de referência em Java são baseados em classes e são usados para criar objetos. Podemos usar inclusive as próprias classes criadas por nós! Aqui estão alguns exemplos comuns de tipos de referência:
	
	\begin{itemize}
		\item \texttt{String}: representa uma sequência de caracteres.
		\item \texttt{Scanner}: usado para receber entrada do usuário.
		\item \texttt{ArrayList}: implementação de uma lista redimensionável.
		\item \texttt{Object}: a superclasse (veremos mais adiante o que é isso) de todos os tipos de objetos em Java.
	\end{itemize}
\end{frame}










