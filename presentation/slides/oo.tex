\begin{frame}
	\frametitle{Visão Geral}
	\begin{itemize}
		\item Introdução à Orientação a Objetos
		\item Classes e Objetos
		\item Encapsulamento
		\item Herança
		\item Polimorfismo
	\end{itemize}
\end{frame}

\begin{frame}[allowframebreaks]
	\frametitle{Classes e Objetos}
	
	\par Uma \textbf{classe} é um modelo para criar objetos que define atributos e métodos. Uma classe também pode ser definida como uma especificação de alguma coisa.
	
	\par Um \textbf{atributo} é uma \textbf{característica} cujo valor pode variar de objeto para objeto.
	
	\par Um \textbf{método} é uma \textbf{ação} que \textbf{pode ou não} variar de objeto para objeto dependendo se tais métodos levam ou não em consideração os valores dos atributos.
	
	\par Já o \textbf{objeto} é aquilo que existe \textbf{real ou virtualmente} segundo a especificação da classe.

	\par \textbf{Exemplos}:
	\par O projeto de uma casa pode ser considerado uma classe já que especifica todas as coisas que a casa fará e todas as propriedades (materiais, isolamento, pisos, etc.) que a casa terá e que, caso desejem, as pessoas que morarão nessa casa poderão ou não modificar. Já o objeto é casa depois de construída de acordo com o projeto: Ela foi construída segundo o seu projeto original, ou seja, segundo sua classe.
\end{frame}

\begin{frame}[fragile]
	\frametitle{Exemplo em Java: Classe e Objeto}
	\par Arquivo \textit{Casa.java}
	\begin{lstlisting}[language=Java]
package org.dedira.oo;

public class Casa {
	public String cor;
	public int qtdeDePisos;
	private int anoDeConstrucao;
	
	public void abrirPorta() {
		System.out.println("Nheeeeeeee.....");
	}

	public int getAnoDeConstrucao() {
		return this.anoDeConstrucao;
	}
	
	public boolean setAnoDeConstrucao(int anoDeConstrucao) {
		
		if (anoDeConstrucao < 0) {
			return false;
		}
		
		this.anoDeConstrucao = anoDeConstrucao;
		return true;
	}
	
	public void exibeInformacoes() {
		System.out.println("Ano:" + anoDeConstrucao + " Cor:" + cor + " Pisos:" + qtdeDePisos);
	}
}
	\end{lstlisting}
\end{frame}

\begin{frame}[fragile]
	\frametitle{Exemplo em Java: Classe e Objeto}
	\par Arquivo \textit{OO.java}
	\begin{lstlisting}[language=Java]
package org.dedira.oo;

public class OO {
	
	public static void main(String[] args) {
		
		Casa minhaCasa = new Casa();
		minhaCasa.setAnoDeConstrucao(2024);
		minhaCasa.cor = "Rosa";
		minhaCasa.qtdeDePisos = 1;
		
		minhaCasa.exibeInformacoes();
	}
}
	\end{lstlisting}
\end{frame}

\begin{frame}
	\frametitle{Exercícios}
	\par Crie um novo programa que represente a entidade trabalhadore. Essa entidade deve ter 10 atributos e 5 acões. Pelo menos 4 das 5 ações devem levar em consideração ou usar pelo menos 5 atributos cada uma.
	
	\par O programa deve rodar devidamente.
\end{frame}